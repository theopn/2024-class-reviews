%%%%%%%%%%%%%%%%%%%%%%%%%%%%%%%%%%%%%%%%%%%%%%%%%%%%%%%%%%%%%%%%%%%%%%%%%%%%%%%%
%%%%%%%%%% Author: Theo Park
%%%%%%%%%%%%%%%%%%%%%%%%%%%%%%%%%%%%%%%%%%%%%%%%%%%%%%%%%%%%%%%%%%%%%%%%%%%%%%%%
\documentclass{report}
\usepackage[utf8]{inputenc}

\usepackage{algorithm} % Wraps algorithmic env and turns it in to a figure -- I prefer not using it bc of the pagebreak
% algorithmic < algorithmicx < algpseudocode < algpseudocodex -- has indentation guide and LComment that others don't
\usepackage{algpseudocodex}
\algnewcommand{\LineComment}[1]{\State \(\triangleright\) #1} % Line comment env
\usepackage{amsmath} % \text
\usepackage{amssymb} % \therefore
\usepackage{forest} % Treeeeeeeee
\usepackage{hyperref} % hyperlink for toc
\usepackage{import} % better \input \include
\usepackage{multicol} % figures side by side
%\usepackage{palatino} % font
\usepackage[default]{sourcesanspro} % font
\usepackage{soul} % \st
\usepackage{tikz} % you know what tikz is
\usetikzlibrary{calc,shapes.multipart,chains,arrows}

\usepackage{minted} % Python code block
\newminted[python]{python}{xleftmargin=\parindent}

% Pkg used for both header and footer
\usepackage{fancyhdr}
% Header
\topmargin=-0.45in
\evensidemargin=0in
\oddsidemargin=0in
\textwidth=6.5in
\textheight=9.0in
\headsep=0.25in

% Footer
\renewcommand{\footrulewidth}{0.5pt} % Footer line thickness
\rfoot{\small{\textit{By Theo Park, based on Purdue Fall 2022 CS251}}}

% Title page
\title{DSA Mini Textbook}
\author{Theo Park}
\date{}

\begin{document}

\maketitle

\pagestyle{fancy}

% TOC

\tableofcontents

% Chapter 1

\chapter{Runtime Analysis}
\import{chapters}{chapter1.tex}

% Chapter 2

\chapter{Intro to Data Structures}
\import{chapters}{chapter2.tex}

% Chapter 3

\chapter{Sorting Algorithms}
\import{chapters}{chapter3.tex}

% Chapter 4

\chapter{Hash Tables}

\section{Division Method}

\section{Multiplication Method}

\section{Collision}

\subsection{Chaining}

\subsection{Open Addressing}

% Chapter 5

\chapter{Search Tree}

\section{Binary Search Tree and Its Limit}

\section{2-3 Tree}

\section{Red-Black Tree}

\section{Left-Leaning Red-Black Tree}

\subsection{Deletion in LLRBT}

% Chapter 6

\chapter{Undirected Graph}

\import{chapters}{chapter6.tex}

% Chapter 7

\chapter{Directed Graphs}

\import{chapters}{chapter7.tex}

% Chapter 8

\chapter{Weighted Graphs}

\import{chapters}{chapter8.tex}

% Chapter 9

\chapter{Strings}

\section{Brute-force String Pattern Matching}

\section{KMP Algorithm}

\section{Trie}

\section{PATRICIA}

\section{Huffman Coding}

\end{document}

