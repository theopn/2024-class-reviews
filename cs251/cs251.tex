%%%%%%%%%%%%%%%%%%%%%%%%%%%%%%%%%%%%%%%%%%%%%%%%%%%%%%%%%%%%%%%%%%%%%%%%%%%%%%%%
%%%%%%%%%% Author: Theo Park
%%%%%%%%%%%%%%%%%%%%%%%%%%%%%%%%%%%%%%%%%%%%%%%%%%%%%%%%%%%%%%%%%%%%%%%%%%%%%%%%
\documentclass{report}
\usepackage[utf8]{inputenc}

\usepackage{algorithm} % Wraps algorithmic env and turns it in to a figure -- I prefer not using it bc of the pagebreak
% algorithmic < algorithmicx < algpseudocode < algpseudocodex -- has indentation guide and LComment that others don't
\usepackage{algpseudocodex}
\algnewcommand{\LineComment}[1]{\State \(\triangleright\) #1} % Line comment env
\usepackage{amsmath} % \text
\usepackage{amssymb} % \therefore
\usepackage{forest} % Treeeeeeeee
\usepackage{hyperref} % hyperlink for toc
\usepackage{import} % better \input \include
\usepackage{multicol} % figures side by side
%\usepackage{palatino} % font
\usepackage[default]{sourcesanspro} % font
\usepackage{soul} % \st
\usepackage{tikz} % you know what tikz is
\usetikzlibrary{calc,shapes.multipart,chains,arrows}

% Pkg used for both header and footer
\usepackage{fancyhdr}
% Header
\topmargin=-0.45in
\evensidemargin=0in
\oddsidemargin=0in
\textwidth=6.5in
\textheight=9.0in
\headsep=0.25in

% Footer
\renewcommand{\footrulewidth}{0.5pt} % Footer line thickness
\rfoot{\small{\textit{By Theo Park, based on Purdue Fall 2022 CS251}}}

% Title page
\title{DSA Mini Textbook}
\author{Theo Park}
\date{}

\begin{document}

\maketitle

\pagestyle{fancy}

% TOC

\tableofcontents

% Preface

\chapter*{Preface}
\addcontentsline{toc}{chapter}{Preface}

% Chapter 1

\chapter{Runtime Analysis}
\import{chapters}{chapter1.tex}

% Chapter 2

\chapter{Intro to Data Structures}
\import{chapters}{chapter2.tex}

% Chapter 3

\chapter{Sorting Algorithms}
\import{chapters}{chapter3.tex}

% Chapter 4

\chapter{Hash Tables}

\section{Division Method}

\section{Multiplication Method}

\section{Collision}

\subsection{Chaining}

\subsection{Open Addressing}

% Chapter 5

\chapter{Search Tree}

\section{Binary Search Tree and Its Limit}

\section{2-3 Tree}

\section{Red-Black Tree}

\section{Left-Leaning Red-Black Tree}

\subsection{Deletion in LLRBT}

% Chapter 6

\chapter{Undirected Graph}

\import{chapters}{chapter6.tex}

% Chapter 7

\chapter{Directed Graphs}

\section{Strong Connectivity}

\subsection{Brute-force Strong Connectivity Algorithm}

\subsection{Brute-force using Stack}

\subsection{Strongly Connected Components and Kosaraju's Algorithm}

\section{Directed Acyclic Graphs}

\subsection{Topological Sort}

% Chapter 8

\chapter{Weighted Graphs}

\section{Shortest Path}

\subsection{Dijkstra's Algorithm}

\noindent \hrulefill
\begin{algorithmic}[1]
  \Function{Dijkstra-shortest-path}{$G, s$} \Comment{$G$ is a graph containing $|V|$ vertices, $s$ is the starting node}
    \State $dist \gets$ array size $|V|$
    \State $prev \gets$ array size $|V|$
    \State $Q \gets$ a new min-heap with distance values as keys
    \item[]
    \LComment{Initialization step}
    \For{$v$ in $Vertices$}
      \If{$v = s$}
        $dist[v] \gets 0$
      \EndIf
      \If{$v \neq s$}
        $dist[v] \gets \infty$
      \EndIf
      \State $prev[u] \gets -1$
      \State Add a tuple $(dist[v], v)$ to $Q$
      \While{$Q$ is not empty}
        \State $u \gets$ the value with minimum dist from $Q$ \Comment{$O(1)$}
      \EndWhile
    \EndFor
  \EndFunction
\end{algorithmic}
\noindent \hrulefill

\subsection{Bellman-Ford Algorithm}

Dijkstra should not be used on a graph with negative edge(s).

\noindent \hrulefill
\begin{algorithmic}[1]
  \Function{Bellman-Ford-shortest-path}{$G, V$} \Comment{$G$ is the graph, $V$ is the vertex list}
    \State $dist \gets$ array size $|V|$
    \State $prev \gets$ array size $|V|$

    \For{$v$ in $V$}
      \If{$v = s$}
        $dist[v] \gets 0$
      \EndIf
      \If{$v \neq s$}
        $dist[v] \gets \infty$
      \EndIf
      \State $prev[u] \gets -1$
    \EndFor
    \For{$i = 1$ to $|V| - 1$}
      \For{$e$ in $E$} \Comment{Edge $e$ connects vertex $u$ and $v$}
        \If{$weight[e]$ + $dist[u]$ $<$ $dist[v]$}
          \State $dist[v] = dist[u] + weight[e]$
          \State $prev[v] = u$
        \EndIf
      \EndFor
    \EndFor
    \LComment{Run the for loop once again. If the shortest distance is updated, then it means there is a negative weight cycle}
    \For{$e$ in $E$} \Comment{Edge $e$ connects vertex $u$ and $v$}
      \If{$weight[e]$ + $dist[u]$ $<$ $dist[v]$}
        \State \Output{Negative weight edge cycle detected}
        \State \Return{}
      \EndIf
    \EndFor
  \EndFunction
\end{algorithmic}
\noindent \hrulefill

\section{Articulation Points}

An articulation point is a vertex such that removing it from the graph increases the number of connected components.

\noindent \hrulefill
\begin{algorithmic}[1]
  \Function{}{$G, s, d$} \Comment{$G$ is the graph, $s$ is the starting vertex, $d$ is the //TODO}
  \State Mark $s$ as visited
  \State $discovery[s] \gets d$
  \State $low[s] \gets d$
  \EndFunction
\end{algorithmic}
\noindent \hrulefill

\section{Minimum Spanning Tree}

\begin{itemize}
  \item Minimum: $\sum weight$ is minimum
  \item Spanning: All vertices in the graph are connected
  \item Tree: No cycle
\end{itemize}

There are two fundamental properties of MST:

\begin{enumerate}
  \item \textit{Cycle Property}: For any cycle $C$ in the graph, if the weight of an edge $e \in C$ is higher than any of individual weights of all other edges in $C$, then its edge cannot belong in the MST.
  \item \textit{Cut Property}: For any \textit{cut} (subdivision of graph with disjoint) $C$ in the graph, if the weight of an edge $e$ in the cut-set of $C$ is strictly smaller than the weights of all other edges of the cut-set of $C$, then this edge belongs to all MST of the graph.
\end{enumerate}

\subsection{Cycle and Cut Properties}

\subsection{Prim's Algorithm}

\section{Union-Find}

\subsection{Kruskal MST Algorithm}

% Chapter 9

\chapter{Strings}

\section{Brute-force String Pattern Matching}

\section{KMP Algorithm}

\section{Trie}

\section{PATRICIA}

\section{Huffman Coding}

\end{document}

